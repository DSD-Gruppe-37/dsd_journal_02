{
  % defines a label prefix local to this file
  % to avoid clashing references when including the same file multiple times
  \newcommand{\labelprefix}{src4-3}

  \subsection{Table Lookup}

  % Intro 
  \subsubsection{Introduktion}

  I denne opgave vil vi bruge en tredje metode til implementering af multiplexing, nemlig et ``lookup table''.
  Vi vil med denne metode implementere sandhedstabellen som vist i opgavens tabel 1.
  Lookup tabellen defineres som en hardcoded konstant array med output værdier, hvorefter den numeriske værdi af inputsne bruges som indeks i arrayet.

  % Design
  \subsubsection{Design og implementering}
  
  Vi har ud fra tabel 1 i opgaven implementeret entity \texttt{table\_lookup} med 3 inputs \texttt{a}, \texttt{b} og \texttt{c}, og outputtet \texttt{x}.
  \CodeRef[\labelprefix]{ex04/vhdl/table_lookup.vhd} viser vores implementering, med den konstante array \texttt{table} der har output værdierne som defineret i sandhedstabelle og type casting af de tre input, først til typen \texttt{unsigned} og derefter til \texttt{integer}, således at værdien kan bruges som indeks.

  \includecode[\labelprefix]{ex04/vhdl/table_lookup.vhd}{Implementering af table lookup i VHDL}{}

  Vi har igen brugt vores testbench entity \texttt{UnitUnderTest} til at teste vores design. Set i \coderef[\labelprefix]{ex04/vhdl/UnitUnderTest.vhd} er vores architecture for test af table lookup entity.

  \includecode[\labelprefix]{ex04/vhdl/UnitUnderTest.vhd}{Test af entity \texttt{table\_lookup}}{linerange={16-17,39-49}}

  % Results
  \subsubsection{Resultater}

  Vores design resulterede efter syntetisering i det RTL-view som er set på \figref{ex4-3_rtl_view}.

  Desuden opstillede vi en funktionel simuleringstest som set på \figref{ex4-3_functional_simulation} hvor alle de 8 mulige kombinationer af inputs og resulterende outputs gennemgås.
  
  \dsdfig{ex4-3_rtl_view}{8cm}{RTL-view af den syntetiserede table lookup}

  \dsdfig{ex4-3_functional_simulation}{\textwidth}{Funktionel simulering af table lookup}

  % Discussion
  \subsubsection{Diskussion}

  Vi ser på \figref{ex4-3_rtl_view} at vores table lookup er blevet syntetiseret til et multiplexer med vores tre input som select og vores hardcodede værdi som data, dette stemmer overens med vores forventning.
  
  Desuden kan vi ud fra vores funktionelle simulation som set på \figref{ex4-3_functional_simulation} se, at sammenhængen mellem input og output stemmer overens med sandhedstabellen. Vi ser desuden at de to outputs som er sat til ``don't care'' (markeret med rødt på billedet) bliver sat til 0 i simulationen.

  % Conclusion
  \subsubsection{Konklusion}

  Vi har i denne opgave erfaret at man som alternativ til \texttt{WITH-SELECT} og \texttt{WHEN-ELSE} også kan bruge en array af konstante værdier og omdanne input til et indeks i arrayet. Dette bliver korrekt omdannet til en multiplexer ligesom de andre selection metoder.
  Vi erfarede også at i den funktionelle simulation bliver ``don't care'' holdt til 0.
}