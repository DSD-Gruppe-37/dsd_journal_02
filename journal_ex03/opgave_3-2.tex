

\subsection{Concatenation}

% Intro 
\subsubsection{Introduktion}
Vi ønsker i denne opgave at benytte VHDL's evne til at konkatinere arrays som f.eks. \texttt{std\_logic\_vector} sammen ved hjælp af \texttt{\&} operatoren.
Dette vil vi bruge til at implementere de logiske operationer ``shift left'', ``shift right'' og ``rotate right''.

% Design
\subsubsection{Design og implementering}

Vi implementerer den givne interface fra opgaven og bruger som beskrevet concatination operatoren \texttt{\&} til at shift og rotate vores input på tre forskellige måder.
Til ``shift left'' tages et udsnit af inputtet og konkatineres med et 0.
Til ``shift right'' tages to 0'er og konkatineres med et udsnit af inputtet.
Til ``rotate right'' tages et udsnit af lavest bits fra inputtet og konkatineres med et udsnit af højeste bits fra inputtet.
Vores implementering ses på \srcref{ex03/vhdl/shift_div.vhd}.

\dsdsrc{ex03/vhdl/shift_div.vhd}
{Entitien der viser de 3 funktioner: \texttt{a\_shl,a\_shr,a\_ror}}
{0}{20}

\includecode[tb_shift]{ex03/vhdl/UnitUnderTest.vhd}{Testbench port mapping}{linerange={59-70}}

Test af vores shift/rotate entity ses på \coderef[tb_shift]{ex03/vhdl/UnitUnderTest.vhd}.
Ved at sende et output til LED'erne har vi mulighed for at få et visuelt feedback, og dermed udføre en række test, som ses i \tabref{shiftdivtest}.

\dsdtab{
\begin{tabular}{p{4cm}p{4cm}p{4cm}}\toprule
 Input & Forventet output & Reelt output \\\midrule
\texttt{%
input:  0b00011110} & \texttt{%
a\_shl: 0b00111100\newline 
a\_shr: 0b00000111\newline 
a\_ror: 0b11000011} & 
\texttt{%
a\_shl: 0b00111100\newline 
a\_shr: 0b00000111\newline 
a\_ror: 0b11000011}   
\end{tabular}
}{Testcase til \texttt{shift\_div} funktionaliteten}{shiftdivtest}

% Results
\subsubsection{Resultater}
Testen i \tabref{shiftdivtest} viste tilfredsstillende resultater, der blev 1 shiftet til venstre, 2 til højre og en rotation på 3 mod højre.

\dsdfig{ex3-2-shiftdivrtl}{9cm}
{RTL view af hele entitien i \srcref{ex03/vhdl/shift_div.vhd} --- det ses at der ingen logiske elementer bruges --- der flyttes blot rundt på interne forbindelser.}


\dsdfig{ex3-2-shiftdivtechmap}{9cm}
{Technology mapping view af hele \texttt{shift\_div} funktionen --- igen ses det at der flyttes blot rundt på interne forbindelser.}


% //TODO Billede af shift metoder.


% Discussion
\subsubsection{Diskussion}

Efter syntesen gennemgik vi RTL mappingen og technology mapping viewet - de viser begge at der ikke bliver brugt et eneste logisk element i denne proces. Der kan altså shiftes og roteres \emph{gratis}\footnote{Dog gælder: $a\cdot 2^n \quad a, n \in \mathbb{Z}$} i VHDL.

% Conclusion
\subsubsection{Konklusion}

Med få linjer kode kan man skabe en \emph{gratis} funktionalitet, forstået på den måde at \textit{ingen} logiske elementer brugt.
