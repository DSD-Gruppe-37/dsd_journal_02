

\subsection{Concatenation}

% Intro 
\subsubsection{Introduktion}
Vi ønsker i denne opgave at benytte et af VHDL's biblioteker til simplificere nogle processerser. Dette gøre ved at bruge \textit{concatenation}. Ved at udnytte funktioner fra \texttt{ieee.numeric\_std.ALL} - specielt dens multiplikations og divisions operatøre, vil vi kunne omarrangere bits på en nem måde.
% Design
\subsubsection{Design og implementering}

Med en viden om at \texttt{0b0001 * 2 = 0b0010 \& 0b0010 * 4 = 0b1000}, og \texttt{0b1000 / 2 = 0b0100 \& 0b0100 / 4 = 0b0001}, skrev vi en kode der på den måde implementerede en bitrearranger, denne ses i \srcref{ex03/vhdl/shift_div.vhd}. Udover bitshifterne, er der også en bitrotater der tager to \textit{slices} af vores \texttt{std\_logic\_vector} og omarrangere dem i en ny \texttt{std\_logic\_vector}.

\dsdsrc{ex03/vhdl/shift_div.vhd}
{Entitien der viser de 3 funktioner: \texttt{a\_shl,a\_shr,a\_ror}}
{0}{20}

\dsdsrc{ex03/vhdl/UnitUnderTest.vhd}{Testbench port mappingen}{59}{70}

Ved at sende et output til LED'erne har vi mulighed for at få et visuelt feedback, og dermed udføre en række test, som ses i \tabref{shiftdivtest}.

\dsdtab{
\begin{tabular}{p{4cm}p{4cm}p{4cm}}\toprule
 Input & Forventet output & Reelt output \\\midrule
\texttt{%
input:  0b00011110} & \texttt{%
a\_shl: 0b00111100\newline 
a\_shr: 0b00000111\newline 
a\_ror: 0b11000011} & 
\texttt{%
a\_shl: 0b00111100\newline 
a\_shr: 0b00000111\newline 
a\_ror: 0b11000011}   
\end{tabular}
}{Testcase til \texttt{shift\_div} funktionaliteten}{shiftdivtest}

% Results
\subsubsection{Resultater}
Testen i \tabref{shiftdivtest} viste tilfredsstillende resultater, der blev 1 shiftet til venstre, 2 til højre og en rotation på 3 mod højre.

\dsdfig{ex3-2-shiftdivrtl}{9cm}
{RTL view af hele entitien i \srcref{ex03/vhdl/shift_div.vhd} --- det ses at der ingen logiske elementer bruges --- der flyttes blot rundt på interne forbindelser.}


\dsdfig{ex3-2-shiftdivtechmap}{9cm}
{Technology mapping view af hele \texttt{shift\_div} funktionen --- igen ses det at der flyttes blot rundt på interne forbindelser.}


% //TODO Billede af shift metoder.


% Discussion
\subsubsection{Diskussion}
% //TODO shift_div diskussion
Efter syntesen gennemgik vi RTL mappingen og technology mapping viewet - de viser begge at der ikke bliver brugt et eneste logisk element i denne proces. Der kan altså shiftes og roteres \emph{gratis}\footnote{Dog gælder: $a\cdot 2^n \quad a, n \in \mathbb{Z}$} i VHDL.

% Conclusion
\subsubsection{Konklusion}

Med få linjer kode kan man skabe en \emph{gratis} funktionalitet, forstået på den måde at \textit{ingen} logiske elementer brugt.

% //TODO shift_div konklusion
