\subsection{Multiplication}

% Intro 
\subsubsection{Introduktion}

% Design
\subsubsection{Design og implementering}




\dsdfig{ex3-3-mult_techmap}{9cm}
{Technology map view der viser \texttt{mult} funktionen.}


\dsdfig{ex3-3-dspblock}{9cm}
{Ved at aktivere DSP Block Balancing, tvinges FGPA'en til at bruge hardware logikelementer --- Her er vist et udsnit af mappingen.}
\dsdsrc{ex03/vhdl/mult.vhd}
{Entitiy der viser de multiplier funktionen: \texttt{mult}}
{0}{20}
\dsdsrc{ex03/vhdl/unitundertest.vhd}
{Testbench for \texttt{mult}}
{65}{73}

% Results
\subsubsection{Resultater}

\dsdtab{
    \begin{tabular}{p{4cm}p{4cm}p{4cm}}\toprule
 Input & Forventet output & Reelt output \\\midrule
\texttt{A: 0b00010100 $\times$\newline B: 0b00111000\phantom{$\times$}} & 
\texttt{0b0000010001100000} &\texttt{0b0000010001100000}
\end{tabular}
}{Testcase til \texttt{mult}}{multiplyoutcome}

\dsdfig{ex3-3-plot.pdf}{9cm}{Plot der viser forholdet mellem bits og logikelementer - dette er en potensieludvikling.}
\footnote{Dog gælder: $a\cdot b \quad a,b \in 2^n \quad n \in \mathbb{Z}$}
% Discussion
\subsubsection{Diskussion}

\dsdtab{
    \begin{tabular}{p{1cm}p{1cm}}\toprule
 Bits & LEs \\\midrule
1 & 1\\
2 & 4\\
4 & 29\\
8 & 96\\
16 & 337\\
32 & 1403
\end{tabular}
}{Forholdet m. bits og logiskeelementer}{logisketabel}



% Conclusion
\subsubsection{Konklusion}
Denne øvelse viste ikke meget om multiplikation, udover at det fungere fremragende på FGPA'en, men viste tilgengæld hvor effektivt FPGA'en kan syntetiseres.