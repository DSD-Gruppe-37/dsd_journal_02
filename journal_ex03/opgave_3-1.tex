\subsection{Signed and Unsigned Arithmetic}

% Intro 
\subsubsection{Introduktion}

Ved at udnytte VHDLs indbyggede arithmetiske funktioner, har vi mulighed for at simplificere four-bit-adderen fra øvelse 2 (Ø2).
Dette vil ske vha.  \textit{casting} fra \texttt{std\_logic\_vector} til \texttt{unsigned} / \texttt{signed}. 

Vi har i denne opgave, skrevet og testet 3 four-bit-adders: signed, unsigned og unsigned w. carry, 

% Design
\subsubsection{Design og implementering}

Vi brugte i Ø2 \texttt{std\_logic\_vector} som input og output - dette vil ske igen. 

Dog bliver både signalet der behandles, midlertidigt lavet om til enten \texttt{signed} eller \texttt{unsigned}.

Dette giver mulighed for at bruge regneoperatorer på en ny måde - der kan adderes, trækkes fra, multipliceres og divideres.

Den fulde entity, med de tilhørende architectures ses i \srcref{ex3/VHDL/fourbitaddersimple.vhd}, det hele er samlet i én fil.

\dsdsrc{ex03/vhdl/fourbitaddersimple.vhd}{Four-bit-adder: architecture: unsigned with carry}{0}{32}

\dsdsrc{ex03/vhdl/fourbitaddersimple.vhd}{Four-bit-adder: architecture: unsigned\_impl}{34}{47}

\dsdsrc{ex03/vhdl/fourbitaddersimple.vhd}{Four-bit-adder: architecture: signed\_impl}{48}{60}

% Results
\subsubsection{Resultater}

% Discussion
\subsubsection{Diskussion}

% Conclusion
\subsubsection{Konklusion}